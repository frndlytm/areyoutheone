\documentclass{article}


\usepackage{algorithm}
\usepackage[noend]{algpseudocode}
  \algrenewcommand\algorithmicthen{}

\usepackage{amsmath}
\usepackage{amsthm}
  \newtheorem{definition}{Definition}[section]
  \newtheorem{problem}{Problem}[section]

\newcommand{\cross}{ \ensuremath{\times} }
\newcommand{\intersect}{ \ensuremath{\cap} }
\newcommand{\union}{ \ensuremath{\cup} }

% Basically comment out the instructions.
\newcommand{\comment}[1]{}


\title{
  \textit{Are You The One?} \\
  Probabilistic Stable Matching
}
\author{
    Christian J. DiMare-Baits \\
    dimare.c@northeastern.edu \\
    Khoury College of Computer Science, \\
    Northeastern University
}

\begin{document}
\maketitle

\comment{
Please submit your final project report as a single document
in PDF format. You should not upload datasets or code; rather,
you should host these files on Dropbox, GitHub, or some other
external website. You should provide links to these external
resources in your report.

Your report should be at least five pages long and cover the
following topics:

\begin{itemize}
  \item What is the problem you were trying to solve and what
        were your results?
  \item What data, if any, did you collect?
  \item What algorithms or methods did you try?
  \item What were the results of your implementation and/or
        experiments and what conclusions do you draw from them?
  \item What milestones in your grade contract did you
        achieve?
  \item For group projects, what did each team member
        contribute?
\end{itemize}

Feel free to include examples, figures, or small snippets of
code, if it helps your explanations. In general, however, as
noted above, the full data and code should not be part of the
report you submit but hosted externally.
}

\section{Introduction}\label{sec:introduction}

\textit{Are You the One?} is a reality television dating show where
a house of approximately 20 contestants are secretly paired up
by a match-making algorithm. Over the course of approximately
10 weeks, the contestants complete various physical and emotional
challenges to win dates in search of their perfect match. Every
week, the contestants form pairs. Additionally each week, the
contestants vote on a couple to send into the "truth booth" which
reveals the answer about the chosen pair. After all pairs are
locked in, the contestants learn how many pairs they guessed
correctly (but not which pairs are correct), and the contestants
must get at least one pair correct every week or else they lose
25\% of the final prize: \$1,000,000.

Every season, there is a clear divide between "head versus heart"
or "strategy versus emotion", and the show emphasizes a healthy
mixture of the two approaches in order for the contestants to win.
The following seeks to understand this game probabilistically and
to make it playable by AI agents.

\begin{itemize}
  \item First, we translate the show's dynamics into a
        playable environment.
  \item Second, we analyze the probabilistic structure of
        the game's action and observation spaces.
  \item Third, we describe a learning agent that develops
        a strategy for winning the game.
  \item Last, we evaluate learning over randomly
        generated games with respect to number of contestants
        and number of guesses.
\end{itemize}


\subsection{Stable Marriage}

The Stable Marriage Problem is a common constraint satisfaction problem
in graph theory that seeks to find pairs of nodes that maximizes the
preferences in a weighted bipartite graph. First, here are some
definitions in graph theory.

\begin{definition}[Weighted Bipartite Graph]
  Let $G = (A, B, E, W)$ be a weighted bipartite graph with node
  set $V = A \union B$, edges $E \in A \cross B$ which cross
  between $A$ and $B$, and a matrix of weights $W$ indexed as
  $W_{a,b}$ containing the weight of the edge between $(a, b) \in E$.

  For consistency in notation, let $a, a^{\prime}, \ldots \in A$
  and similarly $b, b^{\prime}, \ldots \in B$.
\end{definition}

\begin{definition}[Matching]
  A matching in a bipartite graph $B$ is a subset of edges,
  $M \subset E$, such that:

  \begin{itemize}
    \item all vertexes in one of $A$ or $B$ are covered by an edge in $M$, and
    \item no node in $V$ belongs to more than one edge in $M$.
  \end{itemize}

  A perfect matching further restricts the matching so that it
  covers all of $V$.
\end{definition}

Given a weighted bipartite graph, $G$, a stable marriage is an
edge $(a, b) \in E$ such that $W_{a, b} \geq W_{a^\prime, b}$,
$\forall a^\prime \in A$. The objective of the stable marriage
problem is to find this preferential assignment. The most common
algorithm for finding a stable marriage is the Gale-Shapely algorithm
given below.

\begin{algorithm}
  \caption{Gale-Shapely Stable Marriage}\label{alg:gale-shapely}
  \begin{algorithmic}[1]
    \Procedure{Stable-Matching}{A, B, E, W}
    \State Initialize an empty matching, $M$
    \While{$\exists a \in A \And a \notin M$}
    \State $b \gets E.Neighbors(a).SortBy(W).First()$
    \If {$\exists (a^\prime, b) \in M$}
    \If {$W[a, b] > W[a^\prime, b]$}
    \State $M.Remove(a^\prime, b)$
    \State $M.Add(a, b)$
    \EndIf
    \Else
    \State $M.Add(a, b)$
    \EndIf
    \EndWhile
    \Return $M$
    \EndProcedure
  \end{algorithmic}
\end{algorithm}

This algorithm iteratively builds up a matching by adding an edge
to the matching at each step of the while loop, sometimes replacing
an existing edge with a more preferred edge; however, note that the
algorithm really only constrains on the preferences of the second vertex
partition, $B$. Ultimately, this algorithm terminates when all $a \in A$
have chosen a partner, and it is guaranteed to terminate because we are
bounded by a finite set of edges to test. Additionally, this yields the
best possible matching for the elements of $B$, and the worst possible
matching for elements of $A$. In order to counteract this

In the game of \textit{Are You The One>?} We assume that the cast of
contestants were paired up according to this algorithm using some hidden
weight matrix which we hope to approximate as the game progresses.


\subsection{\textit{Are You The One?}}

Given the appropriate foundations in graph theory, we can formally describe
the game of \textit{Are You The One?} with the following dynamics:

\begin{itemize}
  \item The observation space, $S$, is the set of sequences of bipartite
        graphs representing possible matches remaining to be explored, plus
        the number of beams revealed by the previous action. The reason we
        observe a sequence of graphs is to track changes as a result of choices.
  \item The initial observation is the fully connected bipartite graph
        of contestants in the game with 0 beams.
  \item The action space, $A$, contains the set of possible matchings on
        the bipartite graph, crossed with the edge set. This represents a
        guessed match plus a guessed truth booth every week.
  \item The terminal reward $R_h$ is a hidden variable tracking the remaining
        cash prize. Let $@$ be the terminal success state and $R_{max}$ be the
        maximum possible prize pool, then $R(@) = R_h$. A blackout occurs when
        an action yields 0 correct pairs. Whenever a blackout, occurs the prize
        is updated to $R_h = R_h - 0.25 * R_{max}$.
  \item Normally, the game would end when $R_h = 0$; however, this results in
        a sparse reward signal, so we allow the agent to receive negative rewards
        and force it to learn to find perfect matches.
  \item The game terminates when the agent has discovered the random initial
        perfect matching.
\end{itemize}


\section{Results}

\subsection{Milestones}

For the first milestone, I needed to implement a playable game for the
agent to play. This has been accomplished with the implementation of the
AreYouTheOne environmemt deacribed above.

For the second milestone, I needed to construct a brute force search agent
to collect experiences for the agent so I can measure the effect of the
hyper-parameters of the game against the rewards for playing. For example,
how does the number of matches affect the constraint satisfaction problem?
How about the number of weeks? This work was actually performed theoretically
by analyzing the state space of the game with respect to the hyperparameters,
and is described above in the Introduction.

For the third milestone, we needed to implement a probabilistic approach to
solving the game that incorporates match-up ceremonies probabilistically.
The best way we found to approach this was relying on a Naive Bayes approach
that evaluates stable matchings on the basis on the combined likelihoods
that pairs belong or do not belong in the target matching based on the
"beam" dynamics in the game. By maintaining incremental likelihood matrices
for the match versus no match assumptions, we encorporate observations at
each guessing step and seek a stable matching that incorporates them.

Ultimately, we were unable to measure the agent against number of guesses
and prize money to over the course of training on randomly generated
experiences, and missed our final milestone.

\end{document}